\documentclass[letter,twoside,12pt]{article}
\usepackage{lmodern}
\usepackage[T1]{fontenc}
\usepackage[spanish]{babel}
\usepackage[utf8]{inputenc}
\usepackage{amsmath}
 \numberwithin{equation}{section}
\usepackage{amssymb}
\usepackage{amsthm}
\usepackage{fullpage}
\usepackage{latexsym}	
\usepackage{enumerate}
\usepackage{enumitem}
\PassOptionsToPackage{hyphens}{url}\usepackage{hyperref}
\title{Tarea 4}
\newtheorem{theo}{Teorema}
\newtheorem{lemma}[theo]{Lema}
\newtheorem*{defi}{Definición}
\author{Jonathan Andrés Niño Cortés}
\usepackage{amsfonts}
\usepackage{psfrag}
\frenchspacing
\newcommand{\Sp}{\textrm{Sp}}
\newcommand{\Hom}{\textrm{Hom}}
\newcommand{\id}{\textrm{id}}
\newcommand\invisiblesection[1]{%
  \refstepcounter{section}%
  \addcontentsline{toc}{section}{\protect\numberline{\thesection}#1}%
  \sectionmark{#1}}

\begin{document}
\maketitle

\invisiblesection{1}
\begin{enumerate}[label = (\textbf{\arabic*.})]

\item \begin{enumerate}[label = (\textbf{\roman*.})]
\item Sea $ \mathfrak{m} $ una medida exterior de Carathéodory sobre $ X $. Si para todo $ A \subseteq X $ existe $ B \in M_\mathfrak{m} $ con $ A \subseteq B $ y $ \mathfrak{m}(A) = \mathfrak{m}(B) $ la medida exterior se llama regular. Muestre que si $ \mathfrak{m} $ es una medida de Carathéodory regular y $ \mathfrak{m}(A) < \infty $, entonces $ A \in M_\mathfrak{m} $ si y sólo si
\begin{equation}
\mathfrak{m}(A) + \mathfrak{m}(X \backslash A) = \mathfrak{m}(X) \label{eq:med}
\end{equation}

\begin{proof}

Si suponemos que $ A $ es $\mathfrak{m}-$medible entonces por la definición de medibilidad de Carathéodory tenemos que $ \mathfrak{m}(X) = \mathfrak{m}(A) + \mathfrak{m}(X \backslash A) $. Para la otra implicación tómese $A$ tal que cumple la expresión \eqref{eq:med}. Por  regularidad, tenemos que existe $ B \subseteq X $ tal que 
\begin{equation}
 B \in \mathfrak{M}_{\mathfrak{m}}\text{, } A\subseteq B \text{ y  } \mathfrak{m}(A) = \mathfrak{m}(B) \label{eq:reg}
\end{equation}.

Puesto que $B$ es $\mathfrak{m}$-medible tenemos en particular que
\begin{equation}
{\mathfrak{m}}(X) = {\mathfrak{m}}(B) + {\mathfrak{m}}(X\backslash B). \label{eq:medB}
\end{equation}

Utilizando \eqref{eq:med}, \eqref{eq:reg}, \eqref{eq:medB} concluimos que

\begin{equation}
{\mathfrak{m}}(X\backslash A) = {\mathfrak{m}}(X) - {\mathfrak{m}}(A) = {\mathfrak{m}}(X) - {\mathfrak{m}}(B) = {\mathfrak{m}}(X\backslash B). \label{eq:dif}
\end{equation}

Ahora tenemos de nuevo por medibilidad de $B$ que \begin{equation}
{\mathfrak{m}}(X\backslash A) = {\mathfrak{m}}([X\backslash A] \cap B ) + {\mathfrak{m}}([X\backslash A] \cap [X\backslash B] ) \label{eq:medB2}.
\end{equation}

Puesto que $A \subseteq B$ tenemos que $X \backslash B \subseteq X \backslash A $ y por lo tanto $[X\backslash A] \cap [X\backslash B] = [X\backslash B]$, es decir que
\begin{equation}
{\mathfrak{m}}([X\backslash A] \cap [X\backslash B] ) = {\mathfrak{m}}(X\backslash B ) \label{eq:cap}
\end{equation}.

Utilizando \eqref{eq:dif} \eqref{eq:medB2} y \eqref{eq:cap} deducimos que \begin{equation}
{\mathfrak{m}}([X\backslash A] \cap B ) = \mathfrak{m}(X\backslash A) - {\mathfrak{m}}([X\backslash A] \cap [X\backslash B] ) = \mathfrak{m}(X\backslash A) -  \mathfrak{m}(X\backslash B) =  0.
\end{equation} 

Así, concluimos que el conjunto  $ [X\backslash A] \cap B$ tiene medida 0 y por lo tanto es $\mathfrak{m}$-medible. Ver \cite[Teorema 10.7]{hewitt}. Por ultimo, puesto que podemos escribir $ A $ como intersección de $ \mathfrak{m}- $medibles, valiendonos de la siguiente expresión,

\begin{equation}
A = B \cap ([X \backslash A] \cap B)^C
\end{equation}

concluimos que $ A $ es medible.
\end{proof}

\item Muestre que aunque se tenga $ \mathfrak{m}(X) < \infty $ en general pueden existir conjuntos $ A \subseteq X $ \textbf{no medibles} según Carathéodory tales que

$$ \mathfrak{m}(A) + \mathfrak{m}(X \ A) = \mathfrak{m}(X) $$

(Sugerencia: Existe un ejemplo con $ |X| = 3 $.)
\end{enumerate}

\item \begin{enumerate}[label = (\textbf{\roman*.})]
\item Denote por $ \mathbb{R}_d $ la recta real con la topología discreta. Muestre que el espacio $\mathbb{R}_d \times \mathbb{R}  $ con la topología producto es localmente compacto.

\item Para $ f $ definida sobre $ \mathbb{R}_d \times \mathbb{R} $ y $ x \in \mathbb{R} $ fijo, sea $ f_{[x]} $ la función definida sobre $ \mathbb{R} $ por:
$$ f_{[x]}(y):=f(x,y). $$
Muestre que si $ f \in C_{00}(\mathbb{R}_d \times \mathbb{R}) $ se tiene que $ f_{[x]} $ es idénticamente cero excepto que para un número finito de elementos $ x \in \mathbb{R} $.

\item Sea $ S $ la integral de Riemann y defina $ I $ sobre $ C_{00}(\mathbb{R}_d \times \mathbb{R}) $ por:

$$ I(f) := \sum_{x \in \mathbb{R}}S(f_{|x|}). $$

Muestre que $ I $ es un funcional lineal positivo sobre $ C_{00}(\mathbb{R}_d \times \mathbb{R}) $.

\item Sea $ \iota(A)  := \overline{\overline{I}}(\chi_A) $, muestre que el conjunto $ A = \{(x,0) : x \in \mathbb{R}\} $ es localmente $ \iota $–nulo, sin embargo no es $ \iota $–nulo.
\end{enumerate}

\item Sea $ T \subseteq \mathbb{R} $ un conjunto $ \lambda $–medible tal que $ \lambda(T) > 0 $. Muestre que $ T-T $ contiene un intervalo. (Ejercicio 10.43 del libro de texto, viene con sugerencia.)
\end{enumerate}

\bibliography{medida}
\bibliographystyle{alpha}

\end{document}