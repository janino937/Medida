\documentclass[letter,twoside,12pt]{article}
\usepackage{lmodern}
\usepackage[T1]{fontenc}
\usepackage[spanish]{babel}
\usepackage[utf8]{inputenc}
\usepackage{amsmath}
\usepackage{amssymb}
\usepackage{amsthm}
\usepackage{fullpage}
\usepackage{latexsym}
\usepackage{enumerate}
\usepackage{enumitem}
\PassOptionsToPackage{hyphens}{url}\usepackage{hyperref}
\title{Tarea 4}
\newtheorem{theo}{Teorema}
\newtheorem{lemma}[theo]{Lema}
\newtheorem*{defi}{Definición}
\author{Jonathan Andrés Niño Cortés}
\usepackage{amsfonts}
\usepackage{psfrag}
\frenchspacing
\newcommand{\Sp}{\textrm{Sp}}
\newcommand{\Hom}{\textrm{Hom}}
\newcommand{\id}{\textrm{id}}

\begin{document}
\maketitle

\begin{enumerate}[label = (\textbf{\arabic*.})]

\item \begin{enumerate}[label = (\textbf{\roman*.})]
\item Sea $ \mathfrak{m} $ una medida exterior de Carathéodory sobre $ X $. Si para todo $ A \subseteq X $ existe $ B \in M_\mathfrak{m} $ con $ A \subseteq B $ y $ \mathfrak{m}(A) = \mathfrak{m}(B) $ la medida exterior se llama regular. Muestre que si m es una medida de Carathéodory regular y $ \mathfrak{m}(A) < \infty $, entonces $ A \in M_\mathfrak{m} $ si y sólo si
$$ \mathfrak{m}(A) + \mathfrak{m}(X \ A) = \mathfrak{m}(X) $$

\item Muestre que aunque se tenga $ \mathfrak{m}(X) < \infty $ en general pueden existir conjuntos $ A \subseteq X $ \textbf{no medibles} según Carathéodory tales que

$$ \mathfrak{m}(A) + \mathfrak{m}(X \ A) = \mathfrak{m}(X) $$

(Sugerencia: Existe un ejemplo con $ |X| = 3 $.)
\end{enumerate}

\item \begin{enumerate}[label = (\textbf{\roman*.})]
\item Denote por $ \mathbb{R}_d $ la recta real con la topología discreta. Muestre que el espacio $\mathbb{R}_d \times \mathbb{R}  $ con la topología producto es localmente compacto.

\item Para $ f $ definida sobre $ \mathbb{R}_d \times \mathbb{R} $ y $ x \in \mathbb{R} $ fijo, sea $ f_{[x]} $ la función definida sobre $ \mathbb{R} $ por:
$$ f_{[x]}(y):=f(x,y). $$
Muestre que si $ f \in C_{00}(\mathbb{R}_d \times \mathbb{R}) $ se tiene que $ f_{[x]} $ es identicamente cero excepto que para un número finito de elementos $ x \in \mathbb{R} $.

\item Sea $ S $ la integral de Riemann y defina $ I $ sobre $ C_{00}(\mathbb{R}_d \times \mathbb{R}) $ por:

$$ I(f) := \sum_{x \in \mathbb{R}}S(f_{|x|}). $$

Muestre que $ I $ es un funcional lineal positivo sobre $ C_{00}(\mathbb{R}_d \times \mathbb{R}) $.

\item Sea $ \iota(A)  := \overline{\overline{I}}(\chi_A) $, muestre que el conjunto $ A = \{(x,0) : x \in \mathbb{R}\} $ es localmente $ \iota $–nulo, sin embargo no es $ \iota $–nulo.
\end{enumerate}

\item Sea $ T \subseteq \mathbb{R} $ un conjunto $ \lambda $–medible tal que $ \lambda(T) > 0 $. Muestre que $ T-T $ contiene un intervalo. (Ejercicio 10.43 del libro de texto, viene con sugerencia.)
\end{enumerate}

\end{document}