\documentclass[letter,twoside,12pt]{article}
\usepackage{lmodern}
\usepackage[T1]{fontenc}
\usepackage[spanish]{babel}
\usepackage[utf8]{inputenc}
\usepackage{amsmath}
 \numberwithin{equation}{section}
\usepackage{amssymb}
\usepackage{amsthm}
\usepackage{thmtools}
\declaretheorem{theorem}
\declaretheorem[name=Lema, numberwithin=section,refname={lema,lemas},Refname={Lema,Lemas}]{lemma}
\usepackage{fullpage}
\usepackage{latexsym}	
\usepackage{enumerate}
\usepackage{enumitem}
\PassOptionsToPackage{hyphens}{url}
\usepackage{hyperref}
\usepackage{nameref}
\usepackage{cleveref}
\title{Tarea 4}
\newtheorem{theo}{Teorema}
\newtheorem*{defi}{Definición}
\author{Jonathan Andrés Niño Cortés}
\usepackage{amsfonts}
\usepackage{psfrag}
\frenchspacing
\newcommand{\Sp}{\textrm{Sp}}
\newcommand{\Hom}{\textrm{Hom}}
\newcommand{\id}{\textrm{id}}
\newcommand\invisiblesection[1]{%
  \refstepcounter{section}%
  \addcontentsline{toc}{section}{\protect\numberline{\thesection}#1}%
  \sectionmark{#1}}

\begin{document}
\maketitle

\invisiblesection{1}
\begin{enumerate}[label = (\textbf{\arabic*.})]

\item \begin{enumerate}[label = (\textbf{\roman*.})]
\item Sea $ \mathfrak{m} $ una medida exterior de Carathéodory sobre $ X $. Si para todo $ A \subseteq X $ existe $ B \in M_\mathfrak{m} $ con $ A \subseteq B $ y $ \mathfrak{m}(A) = \mathfrak{m}(B) $ la medida exterior se llama regular. Muestre que si $ \mathfrak{m} $ es una medida de Carathéodory regular y $ \mathfrak{m}(A) < \infty $, entonces $ A \in M_\mathfrak{m} $ si y sólo si
\begin{equation}
\mathfrak{m}(A) + \mathfrak{m}(X \backslash A) = \mathfrak{m}(X) \label{eq:med}
\end{equation}

\begin{proof}

Si suponemos que $ A $ es $\mathfrak{m}-$medible entonces por la definición de medibilidad de Carathéodory tenemos que $ \mathfrak{m}(X) = \mathfrak{m}(A) + \mathfrak{m}(X \backslash A) $. Para la otra implicación tómese $A$ tal que cumple la expresión \eqref{eq:med}. Por  regularidad, tenemos que existe $ B \subseteq X $ tal que 
\begin{equation}
 B \in \mathfrak{M}_{\mathfrak{m}}\text{, } A\subseteq B \text{ y  } \mathfrak{m}(A) = \mathfrak{m}(B) \label{eq:reg}
\end{equation}.

Puesto que $B$ es $\mathfrak{m}$-medible tenemos en particular que
\begin{equation}
{\mathfrak{m}}(X) = {\mathfrak{m}}(B) + {\mathfrak{m}}(X\backslash B). \label{eq:medB}
\end{equation}

Utilizando \eqref{eq:med}, \eqref{eq:reg}, \eqref{eq:medB} concluimos que

\begin{equation}
{\mathfrak{m}}(X\backslash A) = {\mathfrak{m}}(X) - {\mathfrak{m}}(A) = {\mathfrak{m}}(X) - {\mathfrak{m}}(B) = {\mathfrak{m}}(X\backslash B). \label{eq:dif}
\end{equation}

Ahora tenemos de nuevo por medibilidad de $B$ que \begin{equation}
{\mathfrak{m}}(X\backslash A) = {\mathfrak{m}}([X\backslash A] \cap B ) + {\mathfrak{m}}([X\backslash A] \cap [X\backslash B] ) \label{eq:medB2}.
\end{equation}

Puesto que $A \subseteq B$ tenemos que $X \backslash B \subseteq X \backslash A $ y por lo tanto $[X\backslash A] \cap [X\backslash B] = [X\backslash B]$, es decir que
\begin{equation}
{\mathfrak{m}}([X\backslash A] \cap [X\backslash B] ) = {\mathfrak{m}}(X\backslash B ) \label{eq:cap}
\end{equation}.

Utilizando \eqref{eq:dif} \eqref{eq:medB2} y \eqref{eq:cap} deducimos que \begin{equation}
{\mathfrak{m}}([X\backslash A] \cap B ) = \mathfrak{m}(X\backslash A) - {\mathfrak{m}}([X\backslash A] \cap [X\backslash B] ) = \mathfrak{m}(X\backslash A) -  \mathfrak{m}(X\backslash B) =  0.
\end{equation} 

Así, concluimos que el conjunto  $ [X\backslash A] \cap B$ tiene medida 0 y por lo tanto es $\mathfrak{m}$-medible. Ver \cite[Teorema 10.7]{hewitt}. Por ultimo, puesto que podemos escribir $ A $ como intersección de $ \mathfrak{m}- $medibles, valiendonos de la siguiente expresión,

\begin{equation}
A = B \cap ([X \backslash A] \cap B)^C
\end{equation}

concluimos que $ A $ es medible.
\end{proof}

\item Muestre que aunque se tenga $ \mathfrak{m}(X) < \infty $ en general pueden existir conjuntos $ A \subseteq X $ \textbf{no medibles} según Carathéodory tales que

$$ \mathfrak{m}(A) + \mathfrak{m}(X \backslash A) = \mathfrak{m}(X) $$

(Sugerencia: Existe un ejemplo con $ |X| = 3 $.)
\begin{proof}
Sea $ X = \{a,b,c\}$, y definamos una medida externa de Carathéodory $\mathfrak{m}: \mathcal{P}(X) \rightarrow \mathbb{R}$ de la siguiente manera:
\begin{eqnarray*}
\mathfrak{m}(\emptyset) = 0
\\ \mathfrak{m}(\{a\}) = \mathfrak{m}(\{b\}) = \mathfrak{m}(\{c\}) = 2
\\ \mathfrak{m}(\{a,b\}) = \mathfrak{m}(\{a,c\}) = \mathfrak{m}(\{b,c\}) = 3
\\ \mathfrak{m}(X)=5.
\end{eqnarray*}

Vemos que se cumple las condiciones de que el vacío es igual a 0 y que la función es positiva. También es facil ver que si $A \subseteq B$ 
entonces $\mathfrak{m}(A) \leq \mathfrak{m}(B)$.

También tenemos subadtividad, observemos que esto es así para algunos casos especificos.
\begin{eqnarray*}
5 = \mathfrak{m}(\{a\})+\mathfrak{m}(\{b,c\}) \geq \mathfrak{m}(X)=5
\\ 4 = \mathfrak{m}(\{a\}) + \mathfrak{m}(\{b\}) \geq \mathfrak{m}(\{a,b\}) = 3
\\ 6 = \mathfrak{m}(\{a\}) + \mathfrak{m}(\{b\}) +\mathfrak{m}(\{c\}) \geq \mathfrak{m}(X) = 5
\\ 6 = \mathfrak{m}(\{a,b\}) + \mathfrak{m}(\{b,c\}) \geq \mathfrak{m}(X) = 5
\end{eqnarray*}

Los demás casos son triviales o analogos a los mostrados anteriormente, por lo que podemos concluir que se tiene subaditividad.

Ahora nótese que $A = \{a\}$ cumple la propiedad \eqref{eq:med}, pues

\begin{equation}
2+3 = \mathfrak{m}(\{a\})+\mathfrak{m}(\{b,c\}) = \mathfrak{m}(X)=5 \nonumber
\end{equation}

Pero $A$ no es medible pues si tomamos $B = \{a,b\} $ tenemos que

\begin{equation}
3 = \mathfrak{m}(B) \lneq \mathfrak{m}(B \cap A) + \mathfrak{m}(B \cap [X\backslash A]) = \mathfrak{m}(\{a\})+\mathfrak{m}(\{b\}) = 2+2=4. \nonumber
\end{equation}
\end{proof}
\end{enumerate}
\pagebreak
\invisiblesection{2}
\item \begin{enumerate}[label = (\textbf{\roman*.})]
\item Denote por $ \mathbb{R}_d $ la recta real con la topología discreta. Muestre que el espacio $\mathbb{R}_d \times \mathbb{R}  $ con la topología producto es localmente compacto.

\begin{proof}

Para demostrar esto primero probamos tres lemas sencillos de topología general.

\begin{lemma}\label{le:dlocal}
Cualquier espacio $X$ con la topología discreta es compacto.
\end{lemma}
\begin{proof}
Tómese cualquier $x \in X $. Si tomamos el singleton $\{x\}$ tenemos que es abierto porque en la topología discreta todo subconjunto de $ X $ es abierto y adicionalmente es compacto porque es finito. Por lo tanto $\{x\}$ es una vecindad compacta de $X$. Puesto que $x$ es arbitrario concluimos que $ X $ es localmente compacto. 
\end{proof}

\begin{lemma} \label{le:rlocal}
El espacio $  \mathbb{R} $ es localmente compacto.
\end{lemma}
\begin{proof}
Tómese cualquier $x \in \mathbb{R} $. Sabemos que existe algún intervalo abierto $ (a,b) $ tal que $ x \in (a,b) $ y puesto que tenemos que $ [a,b] $ es compacto concluimos que $ [a,b] $ es una vecindad compacta de $ x $. Ver \cite[$ \mathsection $ 29, Ejemplo 1]{munkres}.
\end{proof}

\begin{lemma} \label{le:plocal}
Sean $X $ y $ Y$ dos espacio topológicos localmente compactos. Entonces tenemos que el espacio $ X \times Y $ con la topología producto es localmente compacto.
\end{lemma}
\begin{proof}
Tómese un punto cualquiera $ (x,y) $ en el producto. Puesto que $X$ y $Y$ son localmente compactos tenemos que existen conjuntos abiertos $U, V$ y conjuntos compactos $J, K $ tales que $ x \in U \subseteq J \subseteq X $ y $ y \in V \subseteq K \subseteq Y $. Puesto que el producto de compactos es compacto (ver \cite[Teorema 26.7]{munkres}) y el producto de abiertos es abierto (ver \cite[$ \mathsection $ 15, La topología producto de $X \times Y$]{munkres}), concluimos que
\begin{equation*}
(x,y) \in U \times V \subseteq J \times K
\end{equation*}
por lo que concluimos que el espacio es localmente compacto. \end{proof}

Por el \autoref{le:dlocal} y el \autoref{le:rlocal} tenemos que $ \mathbb{R}_d $ y $ \mathbb{R} $ son localmente compactos. Finalmente por el \autoref{le:plocal} concluimos que el producto entre ellos $ \mathbb{R}_d \times \mathbb{R} $ es localmente compacto.
\end{proof}

\item Para $ f $ definida sobre $ \mathbb{R}_d \times \mathbb{R} $ y $ x \in \mathbb{R} $ fijo, sea $ f_{[x]} $ la función definida sobre $ \mathbb{R} $ por:
$$ f_{[x]}(y):=f(x,y). $$
Muestre que si $ f \in C_{00}(\mathbb{R}_d \times \mathbb{R}) $ se tiene que $ f_{[x]} $ es idénticamente cero excepto que para un número finito de elementos $ x \in \mathbb{R} $.
\begin{proof}
Vamos a demostrar otro lema sencillo de topología
\begin{lemma}\label{le:dcompacto}
Sea $ X $ un espacio con la topología discreta. Tenemos que $K$ es compacto si y sólo si $ K $ tiene cardinalidad finita.
\end{lemma}
\begin{proof}
Si $K$ es finito entonces para cualquier cobertura abierta de $K$ puedo seleccionar por cada elemento un conjunto de la cobertura que lo contenga. La colección de estos conjuntos sería una subcobertura finita que contiene a $K$ y por lo tanto $K $ es compacto.

Para la otra implicación tómese un conjunto $A \subseteq X $ de cardinalidad infinita. La colección de singletons de los elementos de $ A $ es una cobertura abierta de $A$ que no tiene subcobertura finita pues si retiramos un solo singleton de la colección dejaríamos de cubrir el elemento correspondiente. Por lo tanto la condición que $K $ sea finito es necesaria para que $K$ sea compacto.
\end{proof}

Tómese cualquier función $f$ con soporte compacto y sea $K$ el conjunto compacto correspondiente. Por un teorema de topología (ver \cite[Teorema 26.5]{munkres}) tenemos que la imagen de $K$ bajo una función continua es compacta. Por lo tanto, si $ \pi_1$ y $\pi_2 $ son las proyecciones de $\mathbb{R}_d \times \mathbb{R} $ a $ \mathbb{R}_d $ y $ \mathbb{R} $ respectivamente tenemos que $ \pi_1(K) $ y $ \pi_2(K) $ son compactos y por lo tanto $ \pi_1(K) \times \pi_2(K) $ (ver nuevamente \cite[Teorema 26.7]{munkres}). Adicionalmente por el \textbf{\autoref{le:dcompacto}} sabemos que $ \pi_1(K)$ es finito. Tenemos que $ K \subseteq \pi_1(K) \times \pi_2(K)$. Luego como $f$ es de soporte compacto tenemos que para cualquier $x \not \in \pi_1(K)$, $f_{[x]}(y)= f(x,y) = 0$ para cualquier $y \in \mathbb{R}$. Así que probamos que $ f_{[x]} $ es idénticamente 0 para cofinitos $x \in \mathbb{R}$.
\end{proof}
\item Sea $ S $ la integral de Riemann y defina $ I $ sobre $ C_{00}(\mathbb{R}_d \times \mathbb{R}) $ por:

$$ I(f) := \sum_{x \in \mathbb{R}}S(f_{[x]}). $$
Muestre que $ I $ es un funcional lineal positivo sobre $ C_{00}(\mathbb{R}_d \times \mathbb{R}) $.

\begin{proof}
Tenemos que $S $ es un funcional lineal, por lo que tenemos que sí $f$ es idénticamente 0 entonces $S(f)=0 $ y además $ S(f+g)= S(f)+S(g)$. Además por el punto anterior tenemos que $f_{[x]}$ es 0 en todos excepto finitos $x \in \mathbb{R}$. Sean $ \{x_1, x_2, \cdots, x_n\} $ los elementos donde $f_{[x]} \not = 0$. Tenemos entonces que

\begin{equation*}
	I(f) = \sum_{x \in \mathbb{R}}S(f_{[x]} ) = \sum_{i=1 }^n S(f_{[x_i]}) = S(\sum_{i=1 }^n f_{[x_i]}).
\end{equation*}

Es fácil ver que para $ f,g \in C_{00}(\mathbb{R}_d \times \mathbb{R})$, $\alpha, \beta \in \mathbb{C} $ 
$$ (\alpha f+\beta g)_{[x]} = \alpha f_{[x]} + \beta g_{[x]}. $$

Luego
\begin{eqnarray*} 
I(\alpha f+\beta g) &=& S(\sum_{i=1}^n \alpha f_{[x]} + \beta g_{[x]}) = S(\alpha\sum_{i=1}^n f_{[x]} + \beta \sum_{i=1}^n g_{[x]})
\\ &=& \alpha S(\sum_{i=1}^n f_{[x]}) + \beta S(\sum_{i=1}^n g_{[x]}) = \alpha I(f)+\beta I(g)
\end{eqnarray*}
\end{proof} por lo cual concluimos que $ I $ es un funcional lineal.

\item Sea $ \iota(A)  := \overline{\overline{I}}(\chi_A) $, muestre que el conjunto $ A = \{(x,0) : x \in \mathbb{R}\} $ es localmente $ \iota $–nulo, sin embargo no es $ \iota $–nulo.

\begin{proof}
Considere los subespacios  de la forma $(\{a\} \times \mathbb{R})$ con $ a \in \mathbb{{R}}_d $. Por un lado tenemos que este subespacio es abierto (de hecho también es un subespacio cerrado) por lo que todo conjunto abierto en el subespacio también es abierto en $\mathbb{R}_d \times \mathbb{R} $.
 Por otra parte este espacio es isomorfo a $\mathbb{R}$ tomando como isomorfismo la proyección sobre $ \mathbb{R} $.
 
 Por ultimo, notese que las funciones $f$ con soporte compacto definidas en este conjunto también son de soporte compacto en todo el espacio si extendemos la función a una función $f'$ que valga 0 en el complemento del subespacio. Esto se da puesto que un compacto en un subespacio también es compacto en el espacio originalm por lo tanto el mismo $K$ se puede usar satisfactoriamente para subespacio y espacio.  También se preserva la continuidad por el lema del pegamiento \cite[Teorema 18.3]{munkres}.
 
 Entonces es fácil ver que para dicho $f'$
 
 $$ I(f') = I(f) = S(f) $$
 
Así que en este subespacio el funcional $ I $ coincide con el funcional $S$ que origina la medida de Lebesgue. Puesto que tenemos la misma medida sobre un espacio que es isomorfo a $\mathbb{R}$ concluimos que para cualquier función $f \in \mathfrak{F}^+$

$$ \overline{\overline{I}}(f') = \overline{\overline{S}}(f) $$  

En particular tenemos que para cualquier conjunto $B \subseteq \mathbb{R}_d \times \mathbb{R} $ y cualquier $a \in \mathbb{R}_d$, 
$$\iota(B \cap (\{a\}) \times \mathbb{R})) = \lambda(\pi_2(B \cap (\{a\}) \times \mathbb{R}))). $$

También tenemos por $\sigma-$aditividad que si un conjunto $B$ es tal que $\pi_1(B)$ es a lo sumo enumerable entonces

 \begin{equation}
 \iota(B) =  \sum_{i=0}^\infty \iota(B \cap (\{a_n\}) \times \mathbb{R})) = \sum_{i=0}^\infty \lambda(\pi_2(B \cap (\{a_n\}) \times \mathbb{R}))), \label{eq:sumcontable}
 \end{equation}
donde $\{a_n\}$ es una enumeración de $\pi_1(B)$.
%Este homeomorfismo también define una biyección $\phi $ entre el espacio de funciones de $(\{a\} \times \mathbb{R}) \to \mathbb{R} $ y el de $\mathbb{R} \to \mathbb{R} $. Explícitamente esta biyección es $ \phi(f)= f_{[a]} $.
%
%Gracias a este homeomorfismo tenemos que todas las propiedades topológicas de los espacios se preservan. Esto nos permite concluir fácilmente que si $f$ es de soporte compacto entonces $\phi(f)$ también lo es. De igual manera si $f$ es semicontinua también lo es $\phi(f)$ y si $f$ es real positiva también lo es $\phi(f)$. Por ultimo obsérvese que por nuestra definición de $I$ si $ f \in C_{00}(\mathbb{R}_d \times \mathbb{R}) $ entonces $I(f) = S(f_{a})$. Todo esto nos permite concluir que para funciones $f$ sobre $(\{a\} \times \mathbb{R}) \to \mathbb{R} $ tenemos que $\overline{\overline{I}}(f)= \overline{\overline{S}}(f_{[a]})$.
%
%Podemos extender una función $f$ definida en $(\{a\} \times \mathbb{R}) $ a $f'$ definida en todo el espacio $(\mathbb{R}_d \times \mathbb{R})$, haciendo que la función tome el valor 0 fuera de los puntos donde esta originalmente definida. Puesto que $(\{a\} \times \mathbb{R}) $ es un subespacio que es abierto y cerrado, es fácil ver que esta extensión preserva las propiedades de continuidad, semicontinuidad, soporte compacto, etc, de las funciones. Por otra parte se sigue de la definición de $I$ que $\overline{\overline{I}}(f') = \overline{\overline{I}}(f)$.
%
%Si $f(x,y) $ es una función tal que $f_{[x]} = 0 $ para todos excepto finitos elementos $ x \in \mathbb{R} $ entonces tenemos
%
%\begin{equation}
%f = \sum_{a \in \mathbb{R}} f'_{a} 
%\end{equation}
%
%donde
%
%\begin{equation}
%f'_{a}(x,y) = 
%\left\{
%	\begin{array}{ll}
%		f_{[a]}(y)  &  \text{si } x = a \\
%		0 & \text{de lo contrario} 
%	\end{array}
%\right. \label{eq:compf}
%\end{equation}
%
%En particular si $f$ es medible entonces por aditividad tenemos que
%
%$$ \overline{\overline{I}}(f) =  \sum_{a \in \mathbb{R}} \overline{\overline{S}}(f_{[a]})  $$
%
%Entre el conjunto de funciones medibles sabemos que están contenidas las funciones características de los conjuntos boreleanos. Con esto es fácil demostrar que el conjunto $A $ es localmente $\iota$-nulo, pues por los literales anteriores tenemos que

Con esto podemos demostrar que $A$ es localmete $ \iota $-nulo, pues para cualquier $F$ compacto se tiene que

$$ \iota(A \cap F) = \sum_{x: (x,0) \in F } \overline{\overline{I}}((\chi_A)_{[x]}) = \sum_{x: (x,0) \in F } \overline{\overline{S}}(\chi_{\{0\}}) = \sum_{x: (x,0) \in F } \lambda(\{0\}) =  0 $$ 
Observese que las sumatorias anteriores son finitas, pues solo finitos $ x $ cumplen que $(x,0) \in F $. Esto se debe a que estos puntos están contenidos en $\pi_1(F) \subseteq \mathbb{R}_d $ que es compacto y por lo tanto finito por el \autoref{le:dcompacto}. Por lo tanto las sumatorias estan bien definidas.

Ahora vamos a demostrar que $\iota(A) = \infty $.

Recuérdese que 
\begin{equation}
\iota(A) = \inf\{\iota(U): U \text{ es abierto y } A \subseteq U \}. \label{eq:abiertos}
\end{equation}
Ver \cite[Teorema 9.24]{hewitt}. Por lo tanto, sea $U$ un abierto cualquiera tal que $A \subseteq U $.

Ahora definamos $ U_x $ como $ (U \cap (\{x\} \times \mathbb{R}))$. Es fácil ver que cada $U_x$ es un abierto y adicionalmente tenemos que 
$$\iota(U_x) = \lambda({\pi_2(U_x)}) > 0, $$ pues la proyección envía abiertos en abiertos y para la medida de Lebesgue tenemos que la medida de cualquier abierto siempre es mayor a cero (esto es fácil de ver pues cualquier abierto contiene algún intervalo $ (a,b) $ y por lo tanto su medida es mayor a $b-a$).

Pero esto no basta para demostrar que la medida de $U$ es infinito. Para esto vamos a probar que existe un $\alpha > 0 $ para el cual existen infinitos $U_x $ tales que $\iota(U)\geq\alpha$

Supongase por contradicción que este no es el caso. Es decir, que para cualquier $\alpha $ existen a lo sumo finitos $x \in \mathbb{R}_d$ tales que $\iota(U_x) \geq \alpha $. Entonces considere la sucesión de conjuntos $B_n = \{x \in \mathbb{R}_d: \iota(U_x) \geq 1/n  \}$. Tenemos que 
$$ \bigcup_{i=1}^\infty B_n = \mathbb{R}_d $$ pues para todo $ x \in \mathbb{R}$, existe un $n$ tal que $ \iota(U_x)) \geq 1/n $. Llegamos a una contradicción pues la unión de los $ B_n $ es a lo sumo enumerable mientras que $ \mathbb{R}_d $ es no enumerable. (De hecho si suponemos que hay a lo sumo enumerables $x$ tales que $U_x \geq \alpha $ también llegamos a la misma contradicción, por lo cual hay no enumerables $x$'s mayores a un $\alpha$).

Finalmente para el $\alpha$ mencionado anteriormente tomamos una subcolección infinita enumerable $\{U_{x_n}\}$ tal que $\iota(U_{x_n})\geq \alpha$. Es claro que esta subcolección es disyunta dos a dos. Definimos la sucesión creciente de conjuntos $\{V_n\}$, como 
$$V_n = \bigcup_{i=1}^n U_{x_n}\text{; y tenemos por $\sigma$-aditivdad que }\iota(V_n) = \iota(\bigcup_{i=1}^n U_{x_n}) = \sum_{i=0}^n \iota(U_{x_n}) \geq n\alpha.$$



Finalmente es claro que $V_1 \subseteq V_2 \subseteq \cdots \subseteq V_n \subseteq \cdots \subseteq U $. Luego tenemos por el teorema 10.13 \cite{hewitt} que
$$ \iota(U) \geq \iota(\bigcup_{i=1}^\infty V_n) = \lim_{n \to \infty} \iota(V_n) \geq \lim_{n \to \infty} \alpha n = \infty. $$ Concluimos finalmente que $\iota(U)= \infty $, y puesto que esto se hizo para $U$ arbitrario que contuviera a $A$ concluimos por \eqref{eq:abiertos} que $\iota(A)= \infty$ y que por lo tanto no es $\iota$-nulo.

%\begin{lemma}
%Sea $ f \in C_{00}(\mathbb{R}_d \times \mathbb{R}) $ y $f'_a $ definido a partir de $ f $ como en \eqref{eq:compf}. Entonces tenemos que $f'_a \in C_{00}(\mathbb{R}_d \times \mathbb{R})$ y
%$$ I(f'_a) = S(f_{[a]})  $$
%\end{lemma}
%\begin{proof}
%
%
%Podemos escribir $f'_a $ de manera equivalente como
%
%\begin{equation}
%f'_{a}(x,y) = 
%\left\{
%	\begin{array}{ll}
%		f(x,y) & x \in \{a\} \times \mathbb{R} \\
%		0 & x \in  (\mathbb{R}_d \backslash\{a\}) \times \mathbb{R}
%	\end{array}
%\right. \label{eq:compf2}
%\end{equation}
%
%Es claro que $f$ y la función constante 0 son continuas y adicionalmente $x \in \{a\} \times \mathbb{R}$ y su complemento son conjuntos cerrados disyuntos. Luego, por el lema del pegamiento (ver \cite[Teorema 18.3]{munkres}) podemos concluir que $f'(a)$ es continua. 
%
%Sea a $ A = \{(x,y) \in \mathbb{R} f(x,y)>0 \}$, por nuestra hipótesis tenemos que existe un compacto $ K $ tal que $A \subseteq K $. Sea $A' = \{f'_a(x,y) \in \mathbb{R} f(x)>0 \}$. Tenemos que $A' = A \cap (\{x\} \times \mathbb{R}) \subseteq A \subseteq K $, por lo que concluimos que $f'_a $ es de soporte compacto.
%
%Finalmente es fácil ver por la definición de $I$ que 
%$$I(f'_a) = S(f_{[a]}) $$
%\end{proof}
%
%\begin{lemma}
%Sea $ f \in \mathfrak{M}^+ $ y $f'_a $ definido a partir de $ f $ como en \eqref{eq:compf}. Entonces tenemos que $f'_a \in \mathfrak{M}^+$ y
%$$ \overline{I}(f'_a) = \overline{S}(f_{[a]})  $$
%\end{lemma}
%\begin{proof}
%
%Considere los conjuntos de la forma $(f'_a)^{-1}(x, \infty) $. Si $x<0 $ entonces $(f'_a)^{-1}(x, \infty) = \mathbb{R_d} \times \mathbb{R} $ que es abierto. Si  $x \geq 0$ entonces tenemos que $(f'_a)^{-1}(x, \infty) = f^{-1}(x, \infty) \cap (\{x\} \times \mathbb{R}) $. Ambos conjuntos son abiertos, el primero debido a que $f$ es semicontinua y el segundo porque es el producto de dos abiertos. Por lo tanto, concluimos que $f'_a$ es semicontinua.
%
%Ahora observese que hay una correspondencia biyectiva entre el conjunto de funciones con soporte compacto menores a $f_{[a]}$ y el conjunto de funciones de soporte compacto menores a $ f'_{a} $.
%
%\begin{eqnarray*}
%\overline{I}(f'_{a})&=& \sup\{I(g): g \leq f'_{a} g \in C _{0}(\mathbb{R}_d \times \mathbb{R}) \} = \sup\{I(g): g \leq f'_{a} g \in C _{00}(\mathbb{R}_d \times \mathbb{R}) \}\\
% &=& \sup\{S(g): g \leq f_{[a]} g \in C _{00}(\mathbb{R}) \} = \overline{S}(f_{[a]}). 
%\end{eqnarray*} 
%\end{proof}

\end{proof}
\end{enumerate}

\invisiblesection{3}
\pagebreak

\item Sea $ T \subseteq \mathbb{R} $ un conjunto $ \lambda $–medible tal que $ \lambda(T) > 0 $. Muestre que $ T-T $ contiene un intervalo. (Ejercicio 10.43 del libro de texto, viene con sugerencia.)
\end{enumerate}

\begin{proof}
Siguiendo los hints propuestos en \cite[Ejercicio 10.43]{hewitt} vamos a demostrar tres lemas intermedios para demostrar el ejercicio.

\begin{lemma}\label{le:int_abierto}
Si $ U = (a,b) $ con $ a, b \in \mathbb{R} $ y $V$ un abierto cualquiera se tiene que la función
	
	\begin{equation*}
		x \mapsto \lambda((x+U) \cap V ) \nonumber
	\end{equation*} es continua. 
\end{lemma}
\begin{proof}
    	Para un $ \epsilon > 0 $ tómese $ \delta = \min(\epsilon,b-a)/2$. Ahora tómese $ x, x' $ tales que $ |x-x'|< \delta $ y asumamos sin perdida de generalidad que $ x' \leq x $ (El caso en que $x \leq x'$ es análogo). Entonces tenemos que 
      \begin{equation*}
      	x +a < \delta + x' + a < b-a +x' + a = b+x' \nonumber
      \end{equation*}
 Y por lo tanto tengo la siguiente cadena de desigualdades.
      \begin{equation*}
      	x'+a \leq x+a < x'+b \leq x+b. \nonumber
      \end{equation*}
Esto me permite partir los conjuntos $x+U$ y $x'+U$ como unión disyunta de dos conjuntos $\lambda$-medibles de la siguiente manera:
\begin{equation*}
      	x+U = (x+a,x+b) = (x+a,x'+b) \cup [x'+b, x+b) \nonumber
      \end{equation*}
      \begin{equation*}
      	x'+U = (x'+a,x'+b) = (x'+a,x+a] \cup (x+a, x'+b) \nonumber
      \end{equation*}
 Entonces por propiedades de la medida tenemos que
% |\lambda((x+U) \cap V)- \lambda((x'+U) \cap V)|
 \begin{eqnarray*}
 & & |\lambda((x+U) \cap V)- \lambda((x'+U) \cap V)|
 \\ &=& |\lambda((x+a,x'+b) \cup [x'+b, x+b) \cap V)- \lambda((x'+a,x+a] \cup (x+a, x'+b) \cap V)|
 \\ &=& |\lambda([x+a, x'+b) \cap V) +\lambda([x'+b, x+b) \cap V)- \lambda((x'+a,x+a] \cap V)
 \\& &-\lambda((x+a, x'+b) \cap V)|
 \\ &=& |\lambda([x'+b, x+b) \cap V)- \lambda((x'+a,x+a] \cap V)|
 \end{eqnarray*}
 Pero por desigualdad triangular
 \begin{eqnarray*}
 & &|\lambda([x'+b, x+b) \cap V)- \lambda((x'+a,x+a] \cap V)|
 \\&\leq & \lambda([x'+b, x+b) \cap V) + \lambda((x'+a,x+a] \cap V)
 \end{eqnarray*}
 Puesto que $[x'+b, x+b) \cap V \subseteq [x'+b, x+b)$ y $(x'+a,x+a] \cap V \subseteq (x'+a,x+a]$ tenemos que 
  \begin{eqnarray*}
 & &\lambda([x'+b, x+b) \cap V) + \lambda((x'+a,x+a] \cap V)
 \\&\leq & \lambda([x'+b, x+b)) + \lambda((x'+a,x+a])
 \\&= & (x+b-(x'+b)) + (x+a-(x'+a))
 \\& = & 2(x-x')
 \\ & < & 2\delta
 \\ & \leq & \epsilon
 \end{eqnarray*}
Concluimos que para este caso la función es continua.\end{proof}

Ahora generalizamos este enunciado un poco más en el siguiente lema.

\begin{lemma} \label{le:abierto}
Para $ U, V \subseteq \mathbb{R}$ dos abiertos cualquiera con $ \lambda(U) < \infty $ se tiene que la función
\begin{equation*}
	x \mapsto \lambda((x+U) \cap V ) \nonumber
\end{equation*} es continua. 
\end{lemma}

\begin{proof}
Tenemos por \cite[Teorema 6.59]{hewitt} que $U = \bigcup_{n = 1}^\infty A_n $ con $\{A_n\}$ una colección enumerable de intervalos abiertos disyuntos y por $\sigma$-aditividad tenemos que

\begin{equation}
\lambda((x+U) \cap V ) = \sum_{n=0}^\infty \lambda((x+A_n) \cap V ) \label{eq:sum}
\end{equation}

Si tomamos $M_n = \lambda(A_n)$ vemos por propiedades de la medida de Lebesgue que 

\begin{equation*}
 \lambda((x+A_n) \cap V ) \leq M_n \nonumber
\end{equation*}

Por otro lado tenemos que

\begin{equation*}
\sum_{n=0}^\infty  M_n = \lambda(U) < \infty \nonumber
\end{equation*}
 
Por lo tanto, podemos concluir por el test $M$ de Weierstrass \cite[Teorema 7.10]{rudin} que \eqref{eq:sum} converge uniformente y por lo demostrado anteriormente en el \textbf{\autoref{le:int_abierto}} tenemos que $ \lambda((x+A_n) \cap V ) $ es continua para todo $n$. Esto a su vez implica que cualquier suma parcial es continua. Concluimos por lo tanto que \eqref{eq:sum} es continua \cite[Teorema 7.12]{rudin}.\end{proof}
A continuación pasamos incluso a un caso más general que el anterior.

\begin{lemma}
Para $ A, B \subseteq \mathbb{R}$ dos conjuntos $ \lambda  $-medibles con $ \lambda(A), \lambda(B)  < \infty $ se tiene que la función
\begin{equation}
	x \mapsto \lambda((x+A) \cap B ) \label{eq:fmed}
\end{equation} es continua. \label{le:medible}
\end{lemma}

\begin{proof}
Por un teorema del libro \cite[Teorema 9.24]{hewitt} podemos encontrar dos conjuntos abiertos $U, V$ tales que $ A \subseteq U $, $\lambda(U) \leq \lambda(A)+\epsilon/6$, $ B \subseteq V $ y $\lambda(V) \leq \lambda(B)+\epsilon/6$.

Tenemos que $ U = A \cup (U \cap A^c)$ y $ V = B \cup (V \cap B^c)$, por lo que tenemos que
\begin{eqnarray*}
\lambda((x+U) \cap V) &=& \lambda((x+A) \cap V) + \lambda((x+((A^c \cap U))) \cap V)
\\&=& \lambda((x+A) \cap B) + \lambda((x+A) \cap (B^c \cap V)) + \lambda((x+((A^c \cap U))) \cap V)
\end{eqnarray*}

Y por lo tanto podemos demostrar la siguiente desigualdad
\begin{equation}
|\lambda((x+U) \cap V) - \lambda((x+A) \cap B)| \leq \lambda(U \cap A^C) + \lambda(V \cap B^C) \leq \epsilon/3 \label{eq:des}
\end{equation}
pues
\begin{eqnarray*}
|\lambda((x+U) \cap V) - \lambda((x+A) \cap B)| & = & |\lambda((x+A) \cap (B^c \cap V)) + \lambda((x+(A^c \cap U)) \cap V)|
\\&\leq & \lambda((x+A) \cap (B^c \cap V)) + \lambda((x+(A^c \cap U)) \cap V)
\\&\leq & \lambda(B^c \cap V) + \lambda(x+(A^c \cap U))
\\&= & \lambda(B^c \cap V) + \lambda(A^c \cap U)
\\&= & \lambda(V) - \lambda(B)  + \lambda(U) - \lambda(A).
\\& \leq & \epsilon/6  + \epsilon/6 = \epsilon/3.
\end{eqnarray*}

Por otra parte por el \textbf{\autoref{le:abierto}} sabemos que existe $\delta $ tal que si $|x-x'| < \delta $ entonces 

\begin{equation}
|\lambda((x+U) \cap V)- \lambda((x'+U) \cap V)| \leq \epsilon/3 \label{eq:contAb} 
\end{equation}

Ahora podemos usar las desigualdades \eqref{eq:des} y \eqref{eq:contAb} para demostrar que si $|x-x'|<\delta $ entonces 

\begin{equation*}
|\lambda((x+A) \cap B)- \lambda((x'+A) \cap B)| \leq \epsilon
\end{equation*}

En efecto tenemos que
\begin{eqnarray*}
& &  |\lambda((x+A) \cap B)- \lambda((x'+A) \cap B)|
\\&\leq&  |\lambda((x+A) \cap B)-\lambda((x+U) \cap V)|+|\lambda((x+U) \cap V)- \lambda((x'+A) \cap B)|
\\&\leq&  |\lambda((x+A) \cap B)-\lambda((x+U) \cap V)|+|\lambda((x+U) \cap V)- \lambda((x'+U) \cap V)|
\\& & + |\lambda((x'+U) \cap V) - \lambda((x'+A) \cap B)|
\\ & \leq & \epsilon/3 + \epsilon/3 + \epsilon/3 = \epsilon.
\end{eqnarray*}

El primer y ultimo valor absoluto se reduce por \eqref{eq:des} y el valor absoluto del medio por \eqref{eq:contAb}.

Por lo tanto, \eqref{eq:fmed} es continua.\end{proof}

Finalmente podemos demostrar el ejercicio. Primero supongase que $ \lambda(T) < \infty $. Podemos usar el lema demostrado anteriormente para concluir que la función $ f(x) = \lambda((x+T)\cap T)$ es continua. Adicionalmente en $x = 0 $, $f(x) = \lambda(T)$. Luego por continuidad en este punto existe un $\delta$ tal que si $ |x| < \delta $ entonces $|f(x)-\lambda(T)| < \lambda(T)/2 $ de donde concluimos que $f(x) > 0 $ pues
\begin{eqnarray*}
|f(x)-\lambda(T)| &<& \lambda(T)/2
\\-\lambda(T)/2 &<& f(x)-\lambda(T)
\\ 0 &<& f(x) -\lambda(T)/2 < f(x)
\end{eqnarray*}

Esto implica claramente que para $x \in (-\delta,\delta)$,  $(x+T)\cap T \not = \emptyset $. Es decir que existe un $t $ tal que $t \in T $ y $t = t'+x $ para algun $t' \in T $. Así que podemos escribir $x = t-t'$, es decir que $ x \in T - T $.
Por lo tanto la bola de radio $\delta $ alrededor de 0 esta contenida en $T - T$ y en particular si tomamos $0<\alpha<\delta$ tenemos que $[-\alpha,\alpha] \subseteq T-T$.

Para el caso en que $T$ tiene medida infinita seleccionamos un conjunto medible $T'$ tal que $T' \subseteq T$ y $ 0<\lambda(T) < \infty $.

Para obtenerlo tómese la secuencia de conjuntos $\{A_n\}$ donde $A_n = [-n,n] \cap T $. Claramente la secuencia es creciente y converge a $ T $ y para cualquier $n$ tenemos que $\lambda(A_n) \leq 2n < \infty $. Por un teorema del libro \cite[Teorema 10.13]{hewitt} tenemos que

\begin{equation}
\lim_{n \to \infty}\lambda(A_n) = \lambda(T) = \infty 
\end{equation}

Por lo que debe existir por lo menos algún $n$ tal que $\lambda(A_n)>0$.

Ahora simplemente tomamos $T' = A_n$, aplicamos lo demostrado anteriormente y puesto que por definición tenemos que $T'-T' \subseteq T-T$ concluimos que existe $\alpha > 0$ tal que $[-\alpha, \alpha] \subseteq T'-T' \subseteq T-T $.
\end{proof}

\bibliography{medida}
\bibliographystyle{alpha}

\end{document}