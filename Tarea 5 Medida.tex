\documentclass[letter,twoside,12pt]{article}
\usepackage{lmodern}

\usepackage{amsmath}
 \numberwithin{equation}{section}
\usepackage{amssymb}
\usepackage{amsthm}
\usepackage{thmtools}\usepackage[T1]{fontenc}
\usepackage[spanish]{babel}
\usepackage[utf8]{inputenc}
\declaretheorem{theorem}
\declaretheorem[name=Lema, numberwithin=section,refname={lema,lemas},Refname={Lema,Lemas}]{lemma}
\usepackage{fullpage}
\usepackage{latexsym}	
\usepackage{enumerate}
\usepackage{enumitem}
\PassOptionsToPackage{hyphens}{url}
\usepackage{hyperref}
\usepackage{nameref}
\usepackage{cleveref}
\title{Tarea 5}
\newtheorem{theo}{Teorema}
\newtheorem*{defi}{Definición}
\author{Jonathan Andrés Niño Cortés}
\usepackage{amsfonts}
\usepackage{psfrag}
\frenchspacing
\newcommand{\Sp}{\textrm{Sp}}
\newcommand{\Hom}{\textrm{Hom}}
\newcommand{\id}{\textrm{id}}
\newcommand\invisiblesection[1]{%
  \refstepcounter{section}%
  \addcontentsline{toc}{section}{\protect\numberline{\thesection}#1}%
  \sectionmark{#1}}

\begin{document}
\maketitle

\invisiblesection{1}
\begin{enumerate}[label = (\textbf{\arabic*.})]
\item Sea $ X $ un espacio topológico, se recuerda que un conjunto $ F_\sigma $ es una unión enumerable de
conjuntos cerrados y un $ G_\delta $ es una intersección enumerable de conjuntos abiertos.
\begin{enumerate}[label = (\textbf{\roman*.})]
\item  Determine si $ \mathbb{Q} $ es un $ F_\sigma $ o un $ G_\delta $ en $ \mathbb{R} $.

\begin{proof}
$ \mathbb{Q} $ es un $ F_\sigma $ pues es enumerable y por lo tanto puede verse como la unión enumerable de singletons que son cerrados en $ \mathbb{R} $.

$\mathbb{Q} $ no es un $G_\delta$. Para demostrar esto necesitamos usar parte del teorema de categorías de Baire.

Un espacio de Baire es un espacio topológico que tiene la siguiente propiedad:

Para cada colección enumerable de conjuntos abiertos densos $ \{U_n\}_{n=1}^\infty $, su intersección $ \bigcap_{n=1}^\infty U_n $ es densa.

\begin{lemma} \label{le:Baire}
$ \mathbb{R} $ es un espacio de Baire. 
\end{lemma}

\begin{proof}

\end{proof}

Ahora para demostrar que $\mathbb{Q} $ no es un $G_\delta$ supongase lo contrario. Entonces existe una colección de abiertos $ U_n $ tal que $ \bigcap_{n=1}^\infty U_n = \mathbb{Q} $. Puesto que cada $ U_n $ contiene a $ \mathbb{Q} $ podemos concluir que $U_n $ es denso.

Denotese por $ \mathbb{I} $ al conjunto de los irracionales.  Puesto que $\mathbb{Q} $ es un $ F_\sigma $ tenemos que su complemento $\mathbb{I}$ es un $G_\delta$ y es igual a la intersección de la colección $\{\mathbb{R}\backslash \{q\}|q \in \mathbb{Q} \} $. Esta colección también resulta ser una colección enumerable de abiertos densos. 

Sin embargo si tomo la intersección de las dos colecciones su resultado es vacío y esto contradice el hecho que $ \mathbb{R} $ es un espacio de Baire.

\end{proof}

\item Escriba un conjunto $ A \in \mathcal{B}(\mathbb{R}) $ que no es ni un $ F_\sigma $ ni un $ G_\delta $ (demuestre por qué).
\begin{proof}
Considere el conjunto $[(-1,1) \cap \mathbb{Q}] \cup [(-1,1)^C \cap \mathbb{I}]$. Claramente este conjunto se encuentra en la $ \sigma $-algebra de Borel de $\mathbb{R}$ pues se obtiene a partir de la intersección, unión y complementos de abiertos, cerrados, $G_\delta$ y $F_\sigma $ todos en la $ \sigma $- álgebra de Borel.

Pero adicionalmente este conjunto cumple la propiedad de no ser un $G_\delta$ ni un $F_\sigma$. 
\end{proof}

\item Muestre que si $ A \in \mathcal{B}(\mathbb{R}) $ entonces $ A $ difiere de ser $ F\sigma $ o $ G\delta $ por un conjunto $ \lambda $–nulo. ($ \lambda $
denota la medida de Lebesgue.)

\begin{proof}
Primero demostramos el siguiente lema

\begin{lemma}\label{le:finito}
Todo conjunto $ \lambda $-medible es $ \sigma $- finito.
\end{lemma}
\begin{proof}
Sea $A$ un conjunto $\lambda $-medible. Tenemos la igualdad
\begin{equation}
A = \bigcup_{n=1}^\infty (A \cap [-n,n])
\end{equation}

Adicionalmente, puesto que para todo $ n $, $ [-n,n] $ es boreleano y por lo tanto $ \lambda $-medible, tenemos que $A \cap [-n,n]$ también lo es y adicionalmente $$ \lambda(A \cap [-n,n]) \leq n < \infty. $$ 
Por lo cual $A$ es la unión enumerable de conjuntos de medida finita. Concluimos que $ A $ es $ \sigma $- finito.
\end{proof}

Gracias al teorema de representación de Riez generalizado \cite[Teorema 12.35]{hewwit} podemos afirmar con certeza que la construcción de la medida de Lebesgue en la sección 9 de \cite{hewwit} es equivalente a la de la sección 12 del mismo libro. 

En vista de lo anterior y del \autoref{le:finito}, es posible utilizar \cite[Teorema 10.34]{hewwit} para concluir que para cualquier conjunto $ \lambda$- medible $A $ existe un boreleano $C$ y un $\sigma$-compacto $F$ tal que $ F \subset A \subset C $ y $ \iota(C \cap F^C) = 0 $.

Pero en la demostración se menciona que $C$ no solo es Boreleano sino que es $ G_\delta $ pues corresponde a la intersección de una secuencia de conjuntos abiertos que contienen a $ A $. Adicionalmente, un conjunto es $\sigma$-compacto si es la unión enumerable de conjuntos compactos. Puesto que en $ \mathbb{R} $ compacto implica cerrado concluimos que $F$ es un $F_\sigma$. Por ultimo, claramente $\iota(C \cap F ^C)$ implica que $\iota(C \cap A ^C) = \iota(A \cap F^C) = 0 $.
 \end{proof}
\end{enumerate}
\invisiblesection{2}
\item Sea $ \mu $ una medida positiva sobre $ X $, sea $ f : X \to [0, \infty] $ medible, $\int_X fd \mu = c $ con $ 0 < c < \infty $
y sea $ \alpha $ una constante. Muestre que:

\begin{equation}
\lim_{n \to \infty} \int_X n\ln(1+(f/n)^\alpha)d\mu = \left\{
	\begin{array}{ll}
		\infty  & \mbox{si } 0 < \alpha < 1 
		 \\ c & \mbox{si } \alpha = 1
		 \\ 0 & \mbox{si } 1 < \alpha < \infty.
	\end{array}
\right. \nonumber
\end{equation}

\begin{proof}
Puesto que $n \ln (1 + (f/n)^\alpha) $ se construye a partir de la suma, potenciación, composición con $ \ln $ que es continua y multiplicación por un escalar concluimos que $n \ln (1 + (f/n)^\alpha) $ es $ \mu $- medible. Ver \cite[Teoremas 11.7 y 11.8]{hewwit}

%Primero debemos cerciorarnos de que la secuencia es creciente
%
%\begin{eqnarray}
% & & (n+1)\ln(1+(f/(n+1))^\alpha)-n\ln(1+(f/n)^\alpha) \nonumber
% \\ &=& n\ln(1+(f/(n+1))^\alpha)+\ln(1+(f/(n+1))^\alpha)-n\ln(1+(f/n)^\alpha) \nonumber
% \\ &=& n\ln(\frac{(n+1)^\alpha+f^\alpha}{(n+1)^\alpha})+\ln(\frac{(n+1)^\alpha+f^\alpha}{(n+1)^\alpha})-n\ln(\frac{n^\alpha+f^\alpha}{n^\alpha}) \nonumber
%  \\ &=& n\ln((n+1)^\alpha+f^\alpha)-n\ln((n+1)^\alpha)+\ln((n+1)^\alpha+f^\alpha) \nonumber
%  \\& & -\ln((n+1)^\alpha)-n\ln(n^\alpha+f^\alpha)+n\ln(n^\alpha) \nonumber
%  \\ &=& n(\ln((n+1)^\alpha+f^\alpha)-\ln((n+1)^\alpha)-\ln(n^\alpha+f^\alpha)+\ln(n^\alpha)) \nonumber
%    \\& & + \ln((n+1)^\alpha+f^\alpha)-\ln((n+1)^\alpha) \nonumber
%     \\ &=& n(\ln((n+1)^\alpha+f^\alpha)+\alpha(\ln(n)-\ln(n+1))-\ln(n^\alpha+f^\alpha)+\ln(n^\alpha)) \nonumber
%    \\& & + \ln((n+1)^\alpha+f^\alpha)-\ln((n+1)^\alpha) \nonumber
%\end{eqnarray} 

Ahora para calcular la expresión que nos interesa enfoquemonos en el siguiente límite:
\begin{equation}
\lim_{x \to \infty} x\ln(1+(f/x)^\alpha) 
= \lim_{x \to \infty} \frac{\ln(1+(f/x)^\alpha)}{\frac{1}{x}} \nonumber
\end{equation}

Cuando $ x \to \infty $ tenemos que $ 1/x \to 0 $ y $ (f/x)^\alpha \to 0 $ lo que implica que $ \ln(1+(f/x)^\alpha) \to \ln(1) = 0 $. Por lo anterior podemos usar la regla de L'Hopital.

\begin{eqnarray}
& & \lim_{x \to \infty} \frac{\ln(1+(f/x)^\alpha)}{\frac{1}{x}} \nonumber
\\ &=& \lim_{x \to \infty} \frac{\frac{1}{1+(f/x)^\alpha}(-\alpha)\frac{f^\alpha}{x^{\alpha+1}}}{-\frac{1}{x^2}} \nonumber
\\ &=& \lim_{x \to \infty} \alpha\frac{1}{\frac{x^{\alpha}+f^\alpha}{x^\alpha}}\frac{f^\alpha}{x^{\alpha-1}} \nonumber
\\ &=& \lim_{x \to \infty} \alpha\frac{x^\alpha}{x^{\alpha}+f^\alpha}\frac{f^\alpha}{x^{\alpha-1}} \nonumber
\\ &=& \lim_{x \to \infty} \alpha\frac{xf^\alpha}{x^{\alpha}+f^\alpha} \nonumber
\\ &=& \lim_{x \to \infty} \alpha\frac{f^\alpha}{x^{\alpha-1}+f^\alpha/x} \nonumber
\end{eqnarray}

Nota: Recuérdese que en el Lema de Fatou el límite se toma de manera puntual. Por esta razón al calcular el límite anterior el $f$ se trata como una constante pues en realidad se esta calculando para cada $a \in X$.

Esta ultima expresión es igual a $f$ si $\alpha = 1 $, 0 si $\alpha > 1$ y $\infty $ si $ \alpha<1. $. Puesto que este límite existe concluimos que nuestro límite inicial es igual, es decir,

\begin{equation}
\lim_{n \to \infty} n\ln(1+(f/n)^\alpha) = \left\{
	\begin{array}{ll}
		\infty  & \mbox{si } 0 < \alpha < 1 
		 \\ f & \mbox{si } \alpha = 1
		 \\ 0 & \mbox{si } 1 < \alpha < \infty.
	\end{array}
\right. \label{eq:lim}
\end{equation}

Si usamos el lema de Fatou $ \cite[12.23]{hewwit} $ sobre la secuencia tenemos que

\begin{equation} \label{eq:fatou}
 \int_x \liminf_{n \to \infty} n\ln(1+(f/n)^\alpha)d\mu \leq \liminf_{n \to \infty} \int_X n\ln(1+(f/n)^\alpha)d\mu
\end{equation}

Entonces para el caso $ 0 < \alpha < \infty $ por la ecuación \eqref{eq:fatou} concluimos que

\begin{eqnarray*}
& &\infty = \int_X \lim_{n \to \infty} n\ln(1+(f/n)^\alpha)d\mu = \int_X \liminf_{n \to \infty} n\ln(1+(f/n)^\alpha)d\mu
\\&\leq&  \liminf_{n \to \infty} \int_X n\ln(1+(f/n)^\alpha)d\mu \leq \lim_{n \to \infty} \int_X n\ln(1+(f/n)^\alpha)d\mu = \infty
\end{eqnarray*} 

Por otra parte, para el caso $ \alpha \geq 1 $ se cumple la siguiente desigualdad.

$$ n\ln((1+(f/n)^\alpha)) \leq n\ln((1+f/n)^\alpha) = \alpha n\ln((1+f/n) $$.

Adicionalmente para todo $x \in \mathbb{R}_{\geq0} $ tenemos lo siguiente:

\begin{equation}
\ln(1+x)\leq x \label{eq:ln}
\end{equation}

Para demostrar esto obsérvese que para $ x=0 $, $ \ln(1) = 0 $.

Por otra parte tenemos que para todo $x \in \mathbb{R}_{\geq0} $

$$ 0 \leq \frac{d}{dx} \ln(1+x) = \frac{1}{1+x} \leq 1 = \frac{d}{dx} x. $$

Esto quiere decir, por una parte que ambas funciones son crecientes para $x > 0$ y por otra que la función $x$ crece más rápido que la función $\ln(1+x)$. Por lo tanto concluimos \eqref{eq:ln}.

Podemos usar \eqref{eq:ln} para el caso $x = f/n $ para concluir que

$$ \ln(1+f/n) \leq f/n. $$

A partir de esta expresión se deduce fácilmente que 

$$ n\ln(1+(f/n)^\alpha) \leq \alpha n\ln(1+f/n)  \leq \alpha f. $$

Por lo tanto, concluimos que la secuencia es dominada por $ \alpha f $ y por lo tanto podemos utilizar el teorema de convergencia dominada \cite[Teorema 12.24]{hewwit} para concluir que 

\begin{equation}
\lim_{n \to \infty} \int_X n \ln(1+(f/n)^\alpha) = \int_X \lim_{n \to \infty}  n\ln(1+(f/n)^\alpha))
\end{equation}

de donde concluimos por \eqref{eq:lim} que si $ \alpha = 1 $ entonces la expresión es igual a $ \int_X f d\mu = c $ y si $ \alpha \geq 1 $ entonces la expresión es igual a $ \int_X 0 d\mu = 0 $ 
\end{proof}
\item  Consiga funciones continuas $ f $ y $ g $ tales que $ f \in \mathcal{L}_2((0, \infty), \mathcal{B}, \lambda) $ pero $ f \not \in \mathcal{L}_1((0, \infty), \mathcal{B}, \lambda) $ y
$ g \in \mathcal{L}_1((0, \infty), \mathcal{B}, \lambda)$ pero  $g \not \in \mathcal{L}_2((0, \infty), \mathcal{B}, \lambda) $
\begin{proof}

Hace falta demostrar que para las funciones elegidas la integral de Lebesgue es equivalente a la integral de Riemann.

Tómese $ f = 1/(x+1) $.

Tenemos que 
\begin{equation}
\int_X \frac{1}{x+1}d\lambda = \int_{0}^\infty \frac{1}{x+1}dx =  \ln(x+1)\Big|_0^\infty = \infty
\end{equation}
Mientras que
\begin{equation}
\int_X \frac{1}{(x+1)^2}d\lambda = \int_{1}^\infty \frac{1}{x^2}dx = -\frac{1}{x}\Big|_0^\infty = 1
\end{equation}

Por lo que concluimos que $ f \in \mathcal{L}_2((0, \infty), \mathcal{B}, \lambda) $ pero $ f \not \in \mathcal{L}_1((0, \infty), \mathcal{B}, \lambda) $

Por otra parte, sea $g$ definida de la siguiente manera:

\begin{equation}
g(x) = \left\{
	\begin{array}{ll}
		1/\sqrt{x}  & \mbox{si } 0 < x \leq 1 
		 \\ 1/x^2 & \mbox{si } x \geq 1
	\end{array}
\right.
\end{equation}
Es fácil comprobar que la función es continua usando el lema del pegamiento. Por otra parte, evaluando las integrales para comprobar si la función pertenece a $ \mathcal{L}_1 $ o $ \mathcal{L}_2 $ tenemos lo siguiente:
\begin{equation}
\int_X |g(x)|d\lambda = \int_{0}^1 \frac{1}{\sqrt{x}}dx + \int_{1}^\infty \frac{1}{x^2}dx =  2\sqrt{x}\Big|_0^1  -\frac{1}{x}\Big|_1^\infty = 3
\end{equation}
\begin{equation}
\int_X |g^2(x)|d\lambda = \int_{0}^1 \frac{1}{x}dx + \int_{1}^\infty \frac{1}{x^4}dx =  \ln(x)\Big|_0^1  -\frac{1}{3}\frac{1}{x^3}\Big|_1^\infty = \infty
\end{equation}

Concluimos que $ g \in \mathcal{L}_1((0, \infty), \mathcal{B}, \lambda)$ pero  $g \not \in \mathcal{L}_2((0, \infty), \mathcal{B}, \lambda) $.
 
\end{proof}
\end{enumerate}
\end{document}