\documentclass[letter,twoside,12pt]{article}
\usepackage{lmodern}
\usepackage[T1]{fontenc}
\usepackage[spanish]{babel}
\usepackage[utf8]{inputenc}
\usepackage{amsmath}
 \numberwithin{equation}{section}
\usepackage{amssymb}
\usepackage{amsthm}
\usepackage{thmtools}
\declaretheorem{theorem}
\declaretheorem[name=Lema, numberwithin=section,refname={lema,lemas},Refname={Lema,Lemas}]{lemma}
\usepackage{fullpage}
\usepackage{latexsym}	
\usepackage{enumerate}
\usepackage{enumitem}
\PassOptionsToPackage{hyphens}{url}
\usepackage{hyperref}
\usepackage{nameref}
\usepackage{cleveref}
\title{Las ecuaciones diferenciales de los procesos de nacimiento y muerte y el problema de momentos de Stieltjes}
\newtheorem{theo}{Teorema}
\newtheorem*{defi}{Definición}
\author{Jonathan Andrés Niño Cortés}
\usepackage{amsfonts}
\usepackage{psfrag}
\frenchspacing
\newcommand{\Sp}{\textrm{Sp}}
\newcommand{\Hom}{\textrm{Hom}}
\newcommand{\id}{\textrm{id}}
\newcommand\invisiblesection[1]{%
  \refstepcounter{section}%
  \addcontentsline{toc}{section}{\protect\numberline{\thesection}#1}%
  \sectionmark{#1}}

\begin{document}
\maketitle

\section{Abstract}

El objetivo del proyecto con Alexander Getmanenko es entender el paper con el mismo nombre que el título escrito por S. Karlin y J. L. McGregor. En este paper se evidencia la fuerte relación entre las dos teorías mencionadas y se dan a conocer resultados obtenidos al evaluar una de las teorías a través de los ojos de la otra. Durante la charla, se empezará hablando sobre los conceptos básicos de los procesos de nacimiento y muerte así como también de la teoría de los problemas de momentos. Por ultimo, se evidenciará el punto en el que ambas teorías se encuentran y una breve descripción de uno de los resultados principales al que se llega en el artículo: El conjunto de problemas de Stieltjes solucionables genera el conjunto de matrices asociadas a procesos de nacimiento y muerte y viceversa.
\cite{chihara_introduction_2011}
\cite{_introduction_1968}
\cite{_introduction_1971}
\cite{karlin_differential_1957}

\textbf{Palabras claves:} Polinomios ortogonales, momentos, medida, procesos de Markoff, matrices infinitas.


   \bibliographystyle{alpha}
    \bibliography{birth}
\end{document}